%----------------------------------------------------------
\subsection*{Задание}\label{blockN.VariantM}
\addcontentsline{toc}{subsection}{Задание}

\textit{@НЕПОСРЕДСТВЕННО ЗДЕСЬ ДОЛЖНА БЫТЬ ПОСТАНОВКА ЗАДАЧИ@}

%----------------------------------------------------------
\subsection*{Цель выполнения лабораторной работы}
\addcontentsline{toc}{subsection}{Цель выполнения лабораторной работы}

Цель выполнения лабораторной работы: \GoalOfResearch.

%----------------------------------------------------------
\subsection{@Название раздела в соответствии с задачей 1@}

\textit{@СОДЕРЖАНИЕ ПОДРАЗДЕЛА@}

% Некоторые примеры оформления ...
%В работе \cite{MIRJALILI201446} авторы пришли к выводу ... $\int_a^{b} f(x)dx$ $\int\limits_a^b f(x)dx$
%
%$$
%A_n = \sum\limits_{i=1}^n \frac{a+b}{(c-d)!}
%$$
%
%\begin{equation}\label{direct.task.chem.kinetics}
%\hat{h}_1(x)=(x-x_1){l_1}^2(x)=(x-0)(1-2x)^2=x(1-4x+4x^2)=4x^3-4x^2+x,
%\end{equation}
%
%Формула \{ \} \eqref{direct.task.chem.kinetics} представляет собой базисный полином Эрмита.
%
%\begin{equation}\label{eq.27.eighteen}
%\hat{h_2}(x)=(x-x_2){l_2}^2(x)=(x-\frac12)(2x)^2=(x-1\frac12)4x^2=4x^3-2x^2.
%\end{equation}
%
%Теперь, подставляя полученные выражения (\ref{eq.27.four})-(\ref{eq.27.seven}), (\ref{hermit}), (\ref{eq.27.eighteen}) в выражение (\ref{eq.27.fourteen}), получается следующее выражение:
%\begin{align*}
%H_{3}(x)&=f(x_1)h_1(x)+f(x_2)h_2(x)+f'(x_2)\hat{h_2}(x)+f'(x_2)\hat{h_2}(x)=\\
%&=1\cdot(16x^3-12x^2+1)+e(12x^2-16x^3)+2\cdot(4x^3-4x^2+x) +2e(4x^3-2x^2)=\\
%&=16x^3-12x^2+1+12ex^2-16ex^3+8x^3-8x^2+2x+8ex^3-4ex^2=\\
%&=8x^3(2-2e+1-e)-4x^2(3-3e+2+e)+2x+1=8(3-e)x^3-4(5-2e)x^2+2x+1.\\
%\end{align*}


%----------------------------------------------------------
\subsection{@Название раздела в соответствии с задачей 2@}

\textit{@СОДЕРЖАНИЕ ПОДРАЗДЕЛА@}

%----------------------------------------------------------
\subsection*{Заключение}
\addcontentsline{toc}{subsection}{Заключение}

\begin{enumerate}
	\item \textit{@Вывод@}
	\item ...
\end{enumerate}

%----------------------------------------------------------
\subsection*{Список использованных источников}

% Уточнить при подготовке материала
\begin{enumerate}
	\item \bibentry{Pershin2018CompMath}
	\item \bibentry{SokolovPershin2021ManLab}
	\item \bibentry{Sokolov2021ManSem}
	\item \bibentry{Pershin2021CompMathTasks}
	\item \bibentry{PershinSokolov2021CompMathLabs}
%	\item ...
\end{enumerate}

%----------------------------------------------------------
\subsection*{Выходные данные}

\textit{\DocOutReference}
%----------------------------------------------------------
% Атрибуты задачи
\labattributes{}{}{}{}{студент группы \group, \Author}{\Year, \Semestr}
%----------------------------------------------------------

