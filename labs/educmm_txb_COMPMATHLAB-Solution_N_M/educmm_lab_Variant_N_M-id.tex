%----------------------------------------------------------
\newcommand{\Title}{Отчет о выполнении лабораторной работы}
\newcommand{\TaskType}{лабораторная работа}
\newcommand{\SubTitle}{по дисциплине <<Вычислительная математика>>}
\newcommand{\LabTitle}{@Тема лабораторной работы@} % Указана в задании, можно немного конкретизировать
\newcommand{\Faculty}{<<Робототехники и комплексной автоматизации>>}
\newcommand{\Department}{<<Системы автоматизированного проектирования (РК-6)>>}
\newcommand{\AuthorFull}{@Фамилия~Имя~Отчество@}
\newcommand{\Author}{@Фамилия~И.О.@}
\newcommand{\group}{@РК6-5XБ@}
\newcommand{\Semestr}{весенний семестр} % Например: осенний семестр или весенний семестр
\newcommand{\BeginYear}{2021}
\newcommand{\Year}{2025}
\newcommand{\Country}{Россия}
\newcommand{\City}{Москва}

% Ключевые слова (представляются для обеспечения потенциальной возможности индексации документа)
\newcommand{\keywordsru}{%
	@keywordsru@} % 5-15 слов или выражений на русском языке, для разделения СЛЕДУЕТ ИСПОЛЬЗОВАТЬ ЗАПЯТЫЕ
\newcommand{\keywordsen}{%
	@keywordsen@} % 5-15 слов или выражений на английском языке, для разделения СЛЕДУЕТ ИСПОЛЬЗОВАТЬ ЗАПЯТЫЕ
% Цель выполнения
\newcommand{\GoalOfResearch}{@цель выполнения лабораторной работы@} % цель исследования (с маленькой буквы и без точки на конце)
%----------------------------------------------------------

